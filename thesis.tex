\documentclass[uplatex, a4paper, 12pt, openany, oneside]{jsbook}

\usepackage[dvipdfmx]{graphicx}
\usepackage[dvipdfmx]{color}
\usepackage[dvipdfmx, bookmarks=true, setpagesize=false]{hyperref}
\usepackage{subcaption} 
\usepackage{pxjahyper}

\usepackage{thesis}
\usepackage{here}
\usepackage{url}

\thesis{修 士 論 文}
\title{
  \centering
    % \scalebox{1.0}{視覚と行動のend-to-end学習により経路追従行動を}\\
    % \scalebox{1.0}{をオンラインで模倣する手法の提案}\\
    % \scalebox{1.0}{-経路選択機能の追加とシナリオに基づく}\\
    % \scalebox{1.0}{目的地へのナビゲーション-}\\
    視覚と行動の end-to-end 学習により経路追従行動をオンラインで模倣する手法の提案\\
-経路選択機能の追加とシナリオに基づく目的地へのナビゲーション-\\
A proposal for an online imitation method of path-tracking behavior by end-to-end learning of vision and action\\
-Adding route selection function and scenario-based navigation to the destination-\\
    % \vspace{-0.3zh}
    % \scalebox{0.7}{}
    % \vspace{-0.6zh}
}
\setlength{\textwidth}{\fullwidth}
\setlength{\evensidemargin}{\oddsidemargin}

\date{2024年 3月提出}
\vspace{-30.0zh}
\teacher{林原 靖男 教授}
\vspace{-15.0zh}
\organization{千葉工業大学 先進工学研究科 未来ロボティクス専攻}
\author{22S1031  春山健太}
\vspace{-15zh}

\renewcommand{\baselinestretch}{1.2}
\begin{document}

%% Front Matter
\frontmatter{}
%
%!TEX root = ../thesis.tex
% \bibliographystyle{plain}
\bibliographystyle{jplain}
% \bibliography{report}
% \nocite{*}
\bibliography{main_bibliography}
%
% \input{backmatter/appendix}
%
%!TEX root = ../thesis.tex
\chapter*{謝辞}
\addcontentsline{toc}{chapter}{謝辞}

本研究を進めるにあたり,熱心にご指導を頂いた林原靖男教授と上田隆一准教授に深く感謝いたします.
屋外自律移動ミーティングや研究相談での厳しいご指導により,研究や技術的,人間的に成長できました.
また,ご指導のおかげで2回の学会発表を行うことができました.
(この修論の有様のように,文章を書く力は疑問が残りますが)

つくばチャレンジを通して実世界でロボットを動かす経験から,
目的に必要な技術をリサーチし,それに基づいて実装する力と,挑戦を恐れず「手を動かして考える」
姿勢を得ることができました.
学部3年で配属されてからの,計4年間の研究室活動では,
LiDARとIMUを破壊するなどの信じられない失態の数々を大変申し訳なく思っています.

研究面での議論やサポート,私生活での
岡田眞也先輩,清岡優祐先輩への感謝は言葉に尽くせません.

また,研究室の同期,後輩(特に屋外自律三人衆)は
日々の生活での研究に関する議論や,息抜きの時間に付き合ってくださり
,ありがとうございました.
そして,学部生の頃より.精神的に辛い時の支えであり,生活に潤いをくれた彼女へ深く感謝いたします.
最後に,大学院を卒業するまでの24年間,私を育ててくださった両親に謝意を表します.



%
%% Main Matter
\mainmatter{}
%
\chapter{序論}
\label{chap:introduction}
%
%\input{introduction/preface}
%
%!TEX root = ../thesis.tex

\section{背景}

\begin{figure}[hbtp]
  \centering
 \includegraphics[keepaspectratio, scale=0.8]
      {images/RaspberryPiMouse.png}
 \caption{Example}
 \label{Fig:Example}
\end{figure}
\newpage

%
% \section{要素技術}
本章では,本研究に関連する要素技術を述べる
\subsection{深層学習}
\subsection{教師あり学習}
\subsection{end-to-end学習}
\subsection{Convolutional Neural Network (CNN)}
\subsection{Long Short Term Memory(LSTM)}
\subsection{メトリックマップに基づくナビゲーション}
ナビゲーションは
本稿では,事前にLiDARやオドメトリなどのセンサを使用して作成したした占有格子地図を用いた目的地までの
自律移動手法をメトリックマッベースの自律移動と呼ぶ.
ROS Navifation stackを用いている.
自己位置推定
自己位置推定ではパーティクルフィルタによって自己位置推定を行うモンテカルロ自己位置推定(MCL)
経路計画
経路計画では占有格子地図からコストマップ ダイクストラ法やA*による
速度指令
本稿では,模倣学習やデータセットの作成に使用している.
\subsection{トポロジカルマップ}
トポロジカルマップは,環境中のランドマークなどの特徴的な箇所(ノード)とその繋がり(エッジ)によって
環境を表現したマップである.
トポロジカルマップをロボットのナビゲーションに適応した研究は〜や〜がある.
本稿でのトポロジカルマップは前述の島田らが提案した形式のものを指す

\subsection{シナリオ}
シナリオはトポロジカルマップ上での,目的地までの経路を
単語の組み合わせで表現したものである,
シナリオは条件と行動で構成される
このシナリオの形式は,人の道案内に関する情報のアンケートによって得られた情報に基づき決定された.


\newpage
\section{関連研究}
メトリックマップを用いず,
カメラ画像に基づいて自律移動を行う研究はいくつかある.
Dhruvら\cite{shah2022lmnav}は,大規模言語モデル(LLM),視覚言語モデル(VLM),
ビジョンナビゲーションモデル(VLM)の3つの大規模モデルを組み合わせ
自然言語による指示から,画像によるナビゲーションを
end-to-endで行う手法を提案している.

miyamotoら\cite{miyamoto}はランドマークとその接続を含むトポロジカルマップと
カメラ画像のセマンティックセグメンテーションを用いた,走行領域検出による
ビジョンベースのナビゲーション手法を提案している.
\begin{figure}[htbp]
    \centering
     \includegraphics[width=120mm]{images/pdf/lmnav.pdf}
     \caption{Embodied instruction following with LM-Nav Quoted from\cite{shah2022lmnav}}
     \label{fig:lmnav}
\end{figure}

\begin{figure}[h]
    \centering
     \includegraphics[height=70mm]{images/pdf/topo_meiji.pdf}
     \caption{Topological map Quoted from\cite{miyamoto}}
     \label{fig:topo_meiji}
\end{figure}
\begin{figure}[h]
    \centering
     \includegraphics[width=120mm]{images/pdf/seg_meiji.pdf}
     \caption{Observation of robot behavior using semantic segmentation Quoted from\cite{miyamoto}}
     \label{fig:seg_meiji}
\end{figure}

これらの手法では,補助的ではあるが,Global Navigation Satellite System(GNSS)や
ホイールオドメトリといった情報を必要としている.
センサ入力という観点で比較すると,本論文で提案するシステムは
カメラ画像のみで目的地まで移動できるという違いがある.
\newpage
\section{目的}
本論文では,カメラ画像のみで目的地に移動するために,
視覚に基づくナビゲーションに対して,
視覚から分岐路の判別,目標方向を提示する機能を追加する.
これにより,カメラ画像とトポロジカルマップから
作成されるシナリオに基づいて,目的地まで自律移動するシステムを構築,提案する.
また,提案するシステムにより
目的地までカメラ入力のみで自律移動できるかを,
実ロボットを用いた実験により検証する.

\section{論文構成}
\ref{chap:introduction}章では本論文における,背景や関連研究,目的を述べた.
\ref{chap:elemental}章では本論文に関連する要素技術を述べる,
\ref{chap:method}章では本論文で提案するシステムについてのべ,
\ref{chap:experiment}章では提案するシステムの有効性を
実ロボットを用いて検証する実験に関して述べる.
\ref{chap:end}章では,本論文の結論と展望を述べる.
%ここにディレクトリのパスを追加していく
\chapter{要素技術}
\label{chap:elemental}
本章では,本研究に関連する要素技術を述べる
\section{深層学習}
\subsection{教師あり学習}
教師あり学習は,機械学習の一種であり,
データセットと呼ばれる
ラベル付きのデータの集合を使用してモデルを訓練する手法である.
データセットには,入力データとそれに対応する正解(ラベル)が含まれている.
モデルはデータセットを利用して,正解となる出力を得られるように学習を行っていく.

\subsection{end-to-end学習}
\ref{fig:e2e}に示すように,入力されるデータから目的の出力を得るために必要なデータを得るために必要な
他段階の処理をニューラルネットワークを用いて直接学習する手法である.
自律移動を例に述べる.
人物や障害物などの物体認識,走行レーンの検出,
経路計画,ステアリング制御などの複数個のタスクを解く必要があるが, 
end-to-end学習では先程のタスクを人間が直接設定せずに
カメラ画像をニューラルネットワークに入力することで,
直接ステアリング操作を学習する.
\begin{figure}[htbp]
    \centering
     \includegraphics[width=90mm]{images/pdf/e2e.pdf}
     \caption{Structure of end-to-end learning}
     \label{fig:e2e}
\end{figure}

\newpage
\section{メトリックマップに基づくナビゲーション}
一般的に,移動ロボットを目的地まで誘導する制御はナビゲーションと呼ばれている.
その中で,事前にLiDARやオドメトリなどのセンサを使用して作成した占有格子地図などの
地図を用いたナビゲーション手法がある.
本稿ではこの地図に基づくナビゲーションを
「メトリックマップに基づくナビゲーション」と呼ぶ.
このメトリックマップに基づくナビゲーションを提供するルールベース制御器として
ROS Navigation stack\cite{ros}がある.
\begin{quote}
    \begin{itemize}
     \item パーティクルフィルタによって自己位置推定を行うモンテカルロ自己位置推定(MCL)
     \item 障害物認識などの局所的または地図全体の大域的なコスト計算と,その結果に基づく経路計画
     \item 経路に追従する並進速度や角速度などの速度指令
    \end{itemize}
   \end{quote}
この中で,経路を追従する行動を「経路追従行動」と呼ぶ.
本論文では,Navigation stackをこの模倣学習やデータセットの作成に使用している.
% \vspace{5zh}
\begin{figure}[htbp]
    \centering
     \includegraphics[width=100mm]{images/pdf/nav.pdf}
     \caption{Path-following module system Quoted from\cite{shimada2020}}
     \label{fig:nav}
\end{figure}

\newpage
\section{トポロジカルマップ}
トポロジカルマップは,環境中のランドマークなどの
特徴的な箇所(ノード)とその繋がり(エッジ)によって
環境を表現したマップである.
島田らは\ref{fig:topo}に示すようなトポロジカルマップの形式\cite{shimada2020}を提案している
トポロジカルマップのノードは,
人の道案内に関するアンケートにおいて,
通路の特徴が多用されていたため,該当する位置に配置され,
その位置がどのような通路の特徴であるかという情報を
持っている.また,「直進」や「右折」といった~で述べるシナリオの行動から
移動するエッジを選択するために,ノードはエッジのIDと相対角度の情報を
持っている.エッジはノードの接続関係を表すように
ノード同士を結んでいる.多くの研究では距離に関する情報を持っていることが
多いが,島田らの形式ではIDのみを持っている.
本稿で用いるトポロジカルマップは,島田らが提案した形式を指す.
\begin{figure}[htbp]
    \centering
     \includegraphics[width=90mm]{images/pdf/topo.pdf}
     \caption{Path-following module system Quoted from\cite{shimada2020}}
     \label{fig:topo}
\end{figure}
\newpage
\section{シナリオ}
シナリオはトポロジカルマップ上での,目的地までの経路を単語の組み合わせで表現したものである,
シナリオは「次の角」や「突き当たり」のような「条件」と「直進」,「右折」のような「行動」の組み合わせにより作成される.
このシナリオの形式は,トポロジカルマップと同様に
人の道案内に関するアンケートで得た,
「次の角まで」のような「条件」と「左折」などの「行動」を組み合わせているという
情報を基に決定している,
例えば,\ref{fig:scenario01}に示すような経路をシナリオで表すと,「3番目の三叉路まで直進.停止」となる,
\begin{figure}[htbp]
    \centering
     \includegraphics[width=90mm]{images/pdf/scenario/scenario01.pdf}
     \caption{Path-following module system Quoted from\cite{shimada2020}}
     \label{fig:scenario01}
\end{figure}
%!TEX root = ../thesis.tex
% \bibliographystyle{plain}
\bibliographystyle{jplain}
% \bibliography{report}
% \nocite{*}
\bibliography{main_bibliography}
%
% \input{backmatter/appendix}
%
%!TEX root = ../thesis.tex
\chapter*{謝辞}
\addcontentsline{toc}{chapter}{謝辞}

本研究を進めるにあたり,熱心にご指導を頂いた林原靖男教授と上田隆一准教授に深く感謝いたします.
屋外自律移動ミーティングや研究相談での厳しいご指導により,研究や技術的,人間的に成長できました.
また,ご指導のおかげで2回の学会発表を行うことができました.
(この修論の有様のように,文章を書く力は疑問が残りますが)

つくばチャレンジを通して実世界でロボットを動かす経験から,
目的に必要な技術をリサーチし,それに基づいて実装する力と,挑戦を恐れず「手を動かして考える」
姿勢を得ることができました.
学部3年で配属されてからの,計4年間の研究室活動では,
LiDARとIMUを破壊するなどの信じられない失態の数々を大変申し訳なく思っています.

研究面での議論やサポート,私生活での
岡田眞也先輩,清岡優祐先輩への感謝は言葉に尽くせません.

また,研究室の同期,後輩(特に屋外自律三人衆)は
日々の生活での研究に関する議論や,息抜きの時間に付き合ってくださり
,ありがとうございました.
そして,学部生の頃より.精神的に辛い時の支えであり,生活に潤いをくれた彼女へ深く感謝いたします.
最後に,大学院を卒業するまでの24年間,私を育ててくださった両親に謝意を表します.



\chapter{実験}
\label{chap:experiment}
%
%\input{introduction/preface}
%
\section{通路分類モジュールの}
\section{実ロボットを用いた実験}
実ロボットを用いて, 提案するシステムにより, ロボットが目的地へ到達可能であるか検証する.
\subsection{実験装置}
実験ではロボットを\ref{fig:gamma}に示す.ロボットはicart-miniをベースに本研究室で開発した
orne gammaを用いる.単眼のウェブカメラを3つ,
2D-LiDAR[(北陽電機 UTM-30LX)]を1つ搭載している. 
制御用の PC には GALLERIA GCR2070RGF-QC-Gを使用している.
メトリックマップベースのナビゲーションには,本学でNavigation stackをもとに開発した
orne navigationを使用する

% \begin{figure}[htbp]
%     \centering
%      \includegraphics[width=100mm]{images/pdf/gamma_sensor.pdf}
%      \caption{Experimental setup}\label{fig:gamma}
% \end{figure}
\subsection{実験方法}
実験環境として\ref{fig:cit3f}に示す千葉工業大学津田沼キャンパス2号館3階の廊下を用いる.
環境中には,三叉路が4つ,角が2つ,突き当たりが2つ含まれている.
経路追従モジュールの訓練および通路分類モジュールのデータセット収集では
〜で示したaからnの経路を順番に走行する.
実験では島田らが用いた50例のシナリオの中から,
〜に示す7例を用いた.
選定の基準は,〜の場所を対象としていること.
ロボットが移動困難な狭い通路が含まれていないこと.
「後ろを向く」など経路追従モジュールができない行動が含まれていないことである.
% \begin{figure}[htbp]
%     \centering
%      \includegraphics[width=100mm]{images/pdf/cit3f.pdf}
%      \caption{Experimental environment}\label{fig:cit3f}
% \end{figure}
% \begin{figure*}[htbp]
%     \centering
%      \includegraphics[width=120mm]{images/pdf/newroute.pdf}
%      \caption{Route used for learning}\label{fig:newroute}
% \end{figure*}
% \begin{figure*}[htbp]
%     \begin{tabular}{ccc}
%         \begin{minipage}[t]{0.3\textwidth}
%             \centering
%             \includegraphics[keepaspectratio, width=57mm]{images/pdf/scenario/scenario01.pdf}
%             \subcaption{Scenario 01}
%             \label{composite}
%         \end{minipage} &
%         \begin{minipage}[t]{0.3\textwidth}
%             \centering
%             \includegraphics[keepaspectratio, width=57mm]{images/pdf/scenario/scenario02.pdf}
%             \subcaption{Scenario 02}
%             \label{Gradation}
%         \end{minipage} &
%         \begin{minipage}[t]{0.3\textwidth}
%             \centering
%             \includegraphics[keepaspectratio, width=57mm]{images/pdf/scenario/scenario03.pdf}
%             \subcaption{Scenario 03}
%             \label{fill}
%         \end{minipage} \\
%         \begin{minipage}[t]{0.3\textwidth}
%             \centering
%             \includegraphics[keepaspectratio, width=57mm]{images/pdf/scenario/scenario04.pdf}
%             \subcaption{Scenario 04}
%             \label{transform}
%         \end{minipage} &
%         \begin{minipage}[t]{0.3\textwidth}
%             \centering
%             \includegraphics[keepaspectratio, width=57mm]{images/pdf/scenario/scenario05.pdf}
%             \subcaption{Scenario 05}
%             \label{image1}
%         \end{minipage} &
%         \begin{minipage}[t]{0.3\textwidth}
%             \centering
%             \includegraphics[keepaspectratio, width=57mm]{images/pdf/scenario/scenario06.pdf}
%             \subcaption{Scenario 06}
%             \label{fig:scenario24}
%         \end{minipage}\\
%         \begin{minipage}[t]{0.3\textwidth}
%             \centering
%             \includegraphics[keepaspectratio, width=57mm]{images/pdf/scenario/scenario07.pdf}
%             \subcaption{Scenario 07}
%             \label{imagess}
%         \end{minipage}
%     \end{tabular}
%     \caption{Scenarios used in the experiment}\label{fig:scenario_exp}
% \end{figure*}

\subsection{実験結果}
シナリオの道順に従い, 三叉路などの分岐路で適切に経路を選択して自律移動する
様子が見られた.結果として,7例すべてでロボットが, 目的地へ到達した.
%

\chapter{おわりに}
\label{chap:end}
\section{結論}
本論文では,
岡田らの従来手法に対し,動的に経路を選択して走行する機能や
視覚に基づいて通路の特徴を検出する機能,分岐路において目的地に向けた進行方向を提示する機能を追加することで,
走行する経路を一定の経路から,設定した任意の目的地に向けた経路へ拡張した.

はじめに経路選択機能の追加を目的として,データセットと学習器の入力へ目標方向を加えた.
そして,追加した機能の有効性をシミュレータを用いた実験により検証した.
実験では,視覚に基づくナビゲーションにおいて,
同一の分岐路であっても目標方向の入力に従い, 
ロボットが適切に経路を選択して移動する様子が見られた.

次に,視覚から通路の特徴を検出する機能,分岐路で目標方向を提示する機能を追加した.
目標方向を提示する機能には,島田らが提案したトポロジカルマップと
「条件」や「行動」による経路の表現(シナリオ)を用いた.
これにより,視覚に基づいて任意の目的地まで移動するシステムを構築した.
そして,実ロボットを用いた実験を行い,構築したシステムにより,カメラ画像とシナリオに基づいて, 
経路を追従し,ロボットが目的地へ到達できることを確認した.

% 事前に作成したメトリックマップを用いず, 
% カメラ画像とシナリオに基づいて経路を追従して目的地まで
% 自律移動するシステムを提案した.
% そして, 実ロボットを用いた実験により提案システムの有効性を検証した.
% 実験では, 提案システムにより,ロボットが目的地へ到達可能であることを
% 確認した.
% \section{今後の展望}
% 本論文では,津田沼キャンパス2号館3階の一部の廊下を対象として実験を行った.
% 今後の展望として,2号館3階のすべての廊下や屋外環境での実験が考えられる.
%% Back Matter
\backmatter{}
%
%!TEX root = ../thesis.tex
% \bibliographystyle{plain}
\bibliographystyle{jplain}
% \bibliography{report}
% \nocite{*}
\bibliography{main_bibliography}
%
% \input{backmatter/appendix}
%
%!TEX root = ../thesis.tex
\chapter*{謝辞}
\addcontentsline{toc}{chapter}{謝辞}

本研究を進めるにあたり,熱心にご指導を頂いた林原靖男教授と上田隆一准教授に深く感謝いたします.
屋外自律移動ミーティングや研究相談での厳しいご指導により,研究や技術的,人間的に成長できました.
また,ご指導のおかげで2回の学会発表を行うことができました.
(この修論の有様のように,文章を書く力は疑問が残りますが)

つくばチャレンジを通して実世界でロボットを動かす経験から,
目的に必要な技術をリサーチし,それに基づいて実装する力と,挑戦を恐れず「手を動かして考える」
姿勢を得ることができました.
学部3年で配属されてからの,計4年間の研究室活動では,
LiDARとIMUを破壊するなどの信じられない失態の数々を大変申し訳なく思っています.

研究面での議論やサポート,私生活での
岡田眞也先輩,清岡優祐先輩への感謝は言葉に尽くせません.

また,研究室の同期,後輩(特に屋外自律三人衆)は
日々の生活での研究に関する議論や,息抜きの時間に付き合ってくださり
,ありがとうございました.
そして,学部生の頃より.精神的に辛い時の支えであり,生活に潤いをくれた彼女へ深く感謝いたします.
最後に,大学院を卒業するまでの24年間,私を育ててくださった両親に謝意を表します.



%

\end{document}
\section{要素技術}
本章では,本研究に関連する要素技術を述べる
\subsection{深層学習}
\subsection{教師あり学習}
\subsection{end-to-end学習}
\subsection{Convolutional Neural Network (CNN)}
\subsection{Long Short Term Memory(LSTM)}
\subsection{メトリックマップに基づくナビゲーション}
ナビゲーションは
本稿では,事前にLiDARやオドメトリなどのセンサを使用して作成したした占有格子地図を用いた目的地までの
自律移動手法をメトリックマッベースの自律移動と呼ぶ.
ROS Navifation stackを用いている.
自己位置推定
自己位置推定ではパーティクルフィルタによって自己位置推定を行うモンテカルロ自己位置推定(MCL)
経路計画
経路計画では占有格子地図からコストマップ ダイクストラ法やA*による
速度指令
本稿では,模倣学習やデータセットの作成に使用している.
\subsection{トポロジカルマップ}
トポロジカルマップは,環境中のランドマークなどの特徴的な箇所(ノード)とその繋がり(エッジ)によって
環境を表現したマップである.
トポロジカルマップをロボットのナビゲーションに適応した研究は〜や〜がある.
本稿でのトポロジカルマップは前述の島田らが提案した形式のものを指す

\subsection{シナリオ}
シナリオはトポロジカルマップ上での,目的地までの経路を
単語の組み合わせで表現したものである,
シナリオは条件と行動で構成される
このシナリオの形式は,人の道案内に関する情報のアンケートによって得られた情報に基づき決定された.


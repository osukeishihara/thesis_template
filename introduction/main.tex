\chapter{序論}
\label{chap:introduction}
%
%\input{introduction/preface}
%
%!TEX root = ../thesis.tex

\section{背景}

\begin{figure}[hbtp]
  \centering
 \includegraphics[keepaspectratio, scale=0.8]
      {images/RaspberryPiMouse.png}
 \caption{Example}
 \label{Fig:Example}
\end{figure}
\newpage

%
% \section{要素技術}
本章では,本研究に関連する要素技術を述べる
\subsection{深層学習}
\subsection{教師あり学習}
\subsection{end-to-end学習}
\subsection{Convolutional Neural Network (CNN)}
\subsection{Long Short Term Memory(LSTM)}
\subsection{メトリックマップに基づくナビゲーション}
ナビゲーションは
本稿では,事前にLiDARやオドメトリなどのセンサを使用して作成したした占有格子地図を用いた目的地までの
自律移動手法をメトリックマッベースの自律移動と呼ぶ.
ROS Navifation stackを用いている.
自己位置推定
自己位置推定ではパーティクルフィルタによって自己位置推定を行うモンテカルロ自己位置推定(MCL)
経路計画
経路計画では占有格子地図からコストマップ ダイクストラ法やA*による
速度指令
本稿では,模倣学習やデータセットの作成に使用している.
\subsection{トポロジカルマップ}
トポロジカルマップは,環境中のランドマークなどの特徴的な箇所(ノード)とその繋がり(エッジ)によって
環境を表現したマップである.
トポロジカルマップをロボットのナビゲーションに適応した研究は〜や〜がある.
本稿でのトポロジカルマップは前述の島田らが提案した形式のものを指す

\subsection{シナリオ}
シナリオはトポロジカルマップ上での,目的地までの経路を
単語の組み合わせで表現したものである,
シナリオは条件と行動で構成される
このシナリオの形式は,人の道案内に関する情報のアンケートによって得られた情報に基づき決定された.


\newpage
\section{関連研究}
メトリックマップを用いず,
カメラ画像に基づいて自律移動を行う研究はいくつかある.
Dhruvら\cite{shah2022lmnav}は,大規模言語モデル(LLM),視覚言語モデル(VLM),
ビジョンナビゲーションモデル(VLM)の3つの大規模モデルを組み合わせ
自然言語による指示から,画像によるナビゲーションを
end-to-endで行う手法を提案している.

miyamotoら\cite{miyamoto}はランドマークとその接続を含むトポロジカルマップと
カメラ画像のセマンティックセグメンテーションを用いた,走行領域検出による
ビジョンベースのナビゲーション手法を提案している.
\begin{figure}[htbp]
    \centering
     \includegraphics[width=120mm]{images/pdf/lmnav.pdf}
     \caption{Embodied instruction following with LM-Nav Quoted from\cite{shah2022lmnav}}
     \label{fig:lmnav}
\end{figure}

\begin{figure}[h]
    \centering
     \includegraphics[height=70mm]{images/pdf/topo_meiji.pdf}
     \caption{Topological map Quoted from\cite{miyamoto}}
     \label{fig:topo_meiji}
\end{figure}
\begin{figure}[h]
    \centering
     \includegraphics[width=120mm]{images/pdf/seg_meiji.pdf}
     \caption{Observation of robot behavior using semantic segmentation Quoted from\cite{miyamoto}}
     \label{fig:seg_meiji}
\end{figure}

これらの手法では,補助的ではあるが,Global Navigation Satellite System(GNSS)や
ホイールオドメトリといった情報を必要としている.
センサ入力という観点で比較すると,本論文で提案するシステムは
カメラ画像のみで目的地まで移動できるという違いがある.
\newpage
\section{目的}
本論文では,岡田らの従来手法に対し,
% カメラ画像のみで目的地に移動するために,
% 視覚に基づくナビゲーションに対して,
視覚から通路の特徴を検出し,目的地に向けた進行方向を提示,動的に経路を選択して移動する機能を追加する.
% これにより,カメラ画像とトポロジカルマップから作成されるシナリオに基づいて,
これにより,視覚に基づいて任意の目的地まで自律移動するシステムを構築する.
また,構築したシステムにより目的地までカメラ入力のみで自律移動できるかを,
実ロボットを用いた実験により検証することを目的とする.

\section{論文構成}
\ref{chap:introduction}章では本論文における,背景や関連研究,目的を述べた.
\ref{chap:elemental}章では本論文に関連する要素技術を述べる.
\ref{chap:path_select}章では,経路選択機能の追加とその機能をシミュレータを用いて確認する実験について述べる.
\ref{chap:scenario_vision}章では,構築するシステムとその有効性を実ロボットを用いて検証する実験に関して述べる.
\ref{chap:end}章では,本論文の結論と展望を述べる.
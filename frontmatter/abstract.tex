%!TEX root = ../thesis.tex
\chapter*{概要}
\thispagestyle{empty}
%
\begin{center}
  \scalebox{1.5}{タイトル}\\
\end{center}
\vspace{1.0zh}
%


キーワード: end-to-end学習 シナリオ ナビゲーション \\
%
 実世界では,ある特定のセンサが機能しない状況におちいり.ロボットの自律移動が継続できない場合がある.
この問題に対しては,複数の種類のセンサを用いて自己位置推定する方法や,
ナビゲーション手段自体を冗長化する方法が考えられる.本研究グループでは,
冗長化に向けて,
一般的に用いられる LiDAR と地 図によるナビゲーションを機械学習で模倣することで,
視覚によるナビゲーションを獲得する方法を提案した.
一般的な模倣学習が人の挙動を模倣するのに対して,
提案手法は LiDARと地図によるナビゲーションの出力を模倣するため,
データセットを収集する手間を省くことができるという特長がある.
さらに前報 [1][2] では,分岐路で指定した方向 (以後, 目標方向と呼 ぶ) に移動する機能を追加した.これにより,ロボット は Fig.1 のように指示された方向に移動するように,カ メラ画像に基づいて経路を移動する.ただし, 前報まで のシステムは,目標方向をカメラ画像により生成して いなかったため, カメラ画像のみで目的地まで移動する ことはできなかった.
本稿では,カメラ画像のみで目的地に移動するために,
カメラ画像から分岐路での目標方向を生成する機能を追加する.
具体的には,島田ら [3] が提案したトポ ロジカルマップと「条件」
や「行動」による経路の表 現(以後,シナリオと呼ぶ)をこれまで提案した手法
へ追加する.これにより,カメラ画像とトポロジカル マップから作成されるシナリオに基づいて,目的地ま で自律移動するシステムを構築する.このシステムに より,事前に作成したメトリックマップを必要せずに,
カメラ画像を入力として目的地まで自律移動できる可能性がある.
本システムはカメラ画像のみで目的地まで移動できるという違いがある.
本稿では,提案するシステムにより目的地まで
カメラ入力のみで自律移動できるかを,
実ロボットを用いた実験により検証する.
\newpage
%%
\chapter*{abstract}
\thispagestyle{empty}
%
\begin{center}
  \scalebox{1.3}{title}
\end{center}
\vspace{1.0zh}
%


keywords: 
